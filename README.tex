\documentclass[12pt]{article}

\usepackage{amssymb, amsmath, amsfonts}
\usepackage{eurosym, geometry, ulem, float}
\usepackage{booktabs, graphicx, tikz, caption, color, setspace, sectsty, comment, footmisc}
\usepackage{natbib, pdflscape, subcaption, array, hyperref}

\normalem

\onehalfspacing
\newtheorem{theorem}{Theorem}
\newtheorem{corollary}[theorem]{Corollary}
\newtheorem{proposition}{Proposition}
\newenvironment{proof}[1][Proof]{\noindent\textbf{#1.} }{\ \rule{0.5em}{0.5em}}

\newtheorem{hyp}{Hypothesis}
\newtheorem{subhyp}{Hypothesis}[hyp]
\renewcommand{\thesubhyp}{\thehyp\alph{subhyp}}

\newcommand{\red}[1]{{\color{red} #1}}
\newcommand{\blue}[1]{{\color{blue} #1}}

\newcolumntype{L}[1]{>{\raggedright\let\newline\\arraybackslash\hspace{0pt}}m{#1}}
\newcolumntype{C}[1]{>{\centering\let\newline\\arraybackslash\hspace{0pt}}m{#1}}
\newcolumntype{R}[1]{>{\raggedleft\let\newline\\arraybackslash\hspace{0pt}}m{#1}}

\geometry{left=1.0in, right=1.0in, top=1.0in, bottom=1.0in}

\begin{document}

\title{Portland: Replication Instruction}
\author{Chanwool Kim\thanks{The Kenneth C. Griffin Department of Economics, The University of Chicago (email: chanwoolkim@uchicago.edu).}}
\date{\today}
\maketitle

\section*{Overview}
README.txt

This document describes the scripts in portland\_servus (in order of the process in the master.R file). For replication, simply run master.R.


DATA PREPARATION
geocoding\_api.R
- Codes Census tract based on address information in the Portland data
tidy\_census\_api.R
- Loads Census data and saves relevant variables
setup\_portland.R
- Loads Portland data and saves in the .RData format (analysis\_info.Rdata for all data, financial\_assistance\_info.RData for financial assistance programme only)
delinquency\_measure.R
- Constructs delinquency-related variables: delinquency status and amount, shutoff status
financial\_assistance\_clean.R
- Constructs variables for various financial assistance programs: financial assistance and payment arrangement
merge\_data.R
- Merge all data, saves account\_info\_analysis.RData

data/analysis/analysis\_info.Rdata contains the raw data
data/analysis/account\_info\_analysis.RData contains the data primarily used in the analysis
- Note: this file is in the account level
- Note: this file does not restrict to residential single family: filter(ACCOUNT\_CLASS\_DFLT \%in\% c("RESSF", "ASST")) to restrict to single family
data/analysis/delinquency\_status/RData contains the panel data on bills and payments




README.txt

This document describes the scripts in portland\_servus (in order of the process in the master.R file). For replication, simply run master.R.


DESCRIPTION

descriptive\_statistics.R
- Create tables for basic descriptive statistics and overview of the data (account types)
descriptive\_statistics\_graph.R
- Create graphs for basic descriptive statistics (pie charts)
descriptive\_statistics\_resmf.R
- Create tables for basic descriptive statistics for residential multi-family units only
descriptive\_statistics\_payment.R
- Create tables/graphs for the payment arrangement programme
- Note: this file also loads financial\_assistance\_info.RData
descriptive\_statistics\_linc.R
- Create tables/graphs for the financial assistance programme
descriptive\_statistics\_did.R
- Create tables for the difference-in-differences exercise



Dropping
\begin{itemize}
	\item Drop any monthly payment bill that occurred at the end of the sample period
\end{itemize}

\end{document}